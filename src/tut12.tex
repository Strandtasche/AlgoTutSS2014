\input{includes/head}
\title[Algorithmen I SS 14]{Tutorium 10}

\usepackage{alltt}

\TitleImage[width=\titleimagewd]{images/title02}
\definecolor{OliveGreen}{RGB}{85,107,47}


\begin{document}

\begin{frame}
  \maketitle
\end{frame}

\begin{frame}{Union-Find Datenstruktur}
	Union-Find ist eine Partition einer Menge so, dass folgende Operationen effizient durchführbar sind:
	\begin{description}
		\item[union(x,y)] Vereinigt die Menge mit x mit der Menge mit y 
		\item[find(x)] Gibt den Repräsentanten der Menge mit x zurück
	\end{description}

	\textbf{Herausfinden ob x in der selben Menge wie y:} find(x) == find(y)

	\ \\
	Repräsentation der Mengen als Bäume: Repräsentanten sind Wurzel
\end{frame}

\begin{frame}{Beispiel Union-Find}
	Zu Beginn: Die Partition besteht aus n ein-elementigen Mengen
	\includegraphics[width=\textwidth]{images/uf01}
	
\end{frame}

\begin{frame}{Beispiel Union-Find}
	union(4,6) \\
	\color{black}find(4) == find(6): \color{OliveGreen}{true} \\
	\color{black}find(1) == find(3): \color{red}{false} \\
	\color{black}find(5) == find(2): \color{red}{false} \\
	\includegraphics[width=\textwidth]{images/uf02}
	
\end{frame}

\begin{frame}{Beispiel Union-Find}
	union(1,3) \\
	\color{black}find(4) == find(6): \color{OliveGreen}{true} \\
	\color{black}find(1) == find(3): \color{OliveGreen}{true} \\
	\color{black}find(5) == find(2): \color{red}{false} \\
	
	\includegraphics[width=\textwidth]{images/uf03}
	
\end{frame}

\begin{frame}{Optimierungen}
	\begin{description}
	\item[find()] Alle traversierte Knoten direkt an die Wurzel gehängt: \\
		$\rightarrow$ Reduziert Baumhöhe. 

	\item[union()] Der kleine Baum wird an den Größeren gehängt
	\end{description}	

\end{frame}

\begin{frame}{Laufzeit}
	\begin{itemize}
		\item Amortisierte Laufzeit pro Operation: $\mathcal{O}(\alpha(n))$
		\begin{itemize}
			\item $\alpha(n)$ ist die Inverse Ackermannfunktion
			\item sehr langsam wachsend:
		\end{itemize}
		\centerline{$\alpha( 2 ^ {2 ^{10 ^{19792}}}) < 5$}
	\end{itemize}
\end{frame}


\end{document}
