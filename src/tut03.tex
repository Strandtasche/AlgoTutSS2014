 \documentclass[c]{beamer}
%\documentclass{beamer}
\listfiles

\mode<presentation>
{
  %\usetheme[deutsch,titlepage0]{KIT}
\usetheme[deutsch]{KIT}
% \usetheme{KIT}

%%  \usefonttheme{structurebold}

  \setbeamercovered{transparent}

  \setbeamertemplate{enumerate items}[circle]
  %\setbeamertemplate{enumerate items}[ball]

}
\usepackage{babel}
\date{}
%\DateText

\newlength{\Ku}
\setlength{\Ku}{1.43375pt}

\usepackage[utf8]{inputenc}
\usepackage[TS1,T1]{fontenc}
\usepackage{array}
\usepackage{multicol}
\usepackage{lipsum}
\usepackage[]{algorithm2e}
\usepackage{amsmath}
\usepackage{color}

\usenavigationsymbols
%\usenavigationsymbols[sfHhdb]
%\usenavigationsymbols[sfhHb]

\subtitle{Algorithmen I SS 14}
\author[]{Lena Winter}

\AuthorTitleSep{\relax}

\institute[ITI]{Institut für Theoretische Informatik}

\TitleImage[width=\titleimagewd]{images/title}

\newlength{\tmplen}

\newcommand{\verysmall}{\fontsize{6pt}{8.6pt}\selectfont}

\title[Algorithmen I SS 14]{Tutorium 1}

\begin{document}
\section{Hashing}
	\subsection{Aufgabe I}
	\begin{frame}{Hashing Aufgabe I}
		Gegeben sei die Hashfunktion: \\
		\begin{center}
			$ H(k) = (k+3) mod 11$ 
		\end{center}
		\ \\
		Mit einer zugehörigen 11 Einträge großen Hashtabelle \\
		\ \\
		Einfügen der Elemente < 70, 71, 66, 97, 82, 15, 93, 40, 19, 76 >
		\\
		\begin{enumerate}
			\item Für Hashing mit verketteten Listen
			\item Für Hashing mit Linearer Suche
		\end{enumerate}

	\end{frame}

	\subsection{Aufgabe I b}	
	\begin{frame}{Hashing Aufgabe Ib)}
		Mit Verketteten Listen:
		\ \\
		\ \\
		\begin{tabular*}{0.75\textwidth}{  c | c | c | c | c | c | c | c | c | c | c  }
			0 & 1 & 2 & 3 & 4 & 5 & 6 & 7 & 8 & 9 & 10 \\
			\hline
			19 & 97 & 76 & 66 &  &  &  & 15 & 93 &  & 40 \\
			  &  &  &  &  &  &  & 70 & 82 &  &  \\
			  &  &  &  &  &  &  &  & 71 &  &  \\
			
		\end{tabular*}
		\ \\
		\parskip 16pt
		Mit Linearer Suche:
		\ \\
		\ \\
		\begin{tabular*}{0.75\textwidth}{  c | c | c | c | c | c | c | c | c | c | c  }
			0 & 1 & 2 & 3 & 4 & 5 & 6 & 7 & 8 & 9 & 10 \\
			\hline
			93 & 97 & 40 & 66 & 19 & 76 &  & 70 & 71 & 82 & 15 \\			
		\end{tabular*}

		
	\end{frame}

\end{document}
