 \documentclass[c]{beamer}
%\documentclass{beamer}
\listfiles

\mode<presentation>
{
  %\usetheme[deutsch,titlepage0]{KIT}
\usetheme[deutsch]{KIT}
% \usetheme{KIT}

%%  \usefonttheme{structurebold}

  \setbeamercovered{transparent}

  \setbeamertemplate{enumerate items}[circle]
  %\setbeamertemplate{enumerate items}[ball]

}
\usepackage{babel}
\date{}
%\DateText

\newlength{\Ku}
\setlength{\Ku}{1.43375pt}

\usepackage[utf8]{inputenc}
\usepackage[TS1,T1]{fontenc}
\usepackage{array}
\usepackage{multicol}
\usepackage{lipsum}
\usepackage[]{algorithm2e}
\usepackage{amsmath}
\usepackage{color}

\usenavigationsymbols
%\usenavigationsymbols[sfHhdb]
%\usenavigationsymbols[sfhHb]

\subtitle{Algorithmen I SS 14}
\author[]{Lena Winter}

\AuthorTitleSep{\relax}

\institute[ITI]{Institut für Theoretische Informatik}

\TitleImage[width=\titleimagewd]{images/title}

\newlength{\tmplen}

\newcommand{\verysmall}{\fontsize{6pt}{8.6pt}\selectfont}

\title[Algorithmen I SS 14]{Tutorium 12}

\usepackage{alltt}

\TitleImage[width=\titleimagewd]{images/alch}

\begin{document}

\begin{frame}
  \maketitle
\end{frame}

\begin{frame}{Greedy Algorithmen}
	\begin{itemize}
		\item Global optimale Lösungen bestehen oft aus lokal optimalen Lösungen
		\item Wähle für die \emph{aktuelle Situation} die optimale Entscheidung.
		\item Keine Reversion von vergangenen Entscheidungen
		\item $\Rightarrow$ Einfach und schnell
	\end{itemize}
\end{frame}

\begin{frame}{Beispiele}
	\large{Greedy Algorithmen die optimale Ergebnisse liefern:}
	\begin{itemize}
		\item Dijkstra für kürzeste Wege
		\item Beide MST Algorithmen
		\begin{itemize}
			\item Kruskal
			\item Jarnik-Prim
		\end{itemize}
		\item Selection Sort
	\end{itemize}

\end{frame}

\begin{frame}
	\begin{center}
		\Huge
		Dynamische Programmierung
	\end{center}
\end{frame}

\begin{frame}{Dynamische Programmierung}
	\begin{block}{Definition}
		Dynamische Programmierung ist ein algorithmisches Muster, das vor allem zur Lösung von Optimierungproblemen eingesetzt wird.
	\end{block}
	\textbf{Idee:}
	\begin{itemize}
		\item Problem in Teilprobleme zerlegen
		\item Lösungen der Teilprobleme speichern
		\item Endlösung aus Lösungen der Teilproblemen zusammensetzten
	\end{itemize}
\end{frame}

\begin{frame}{Dynamische Programmierung: Zwischenspeichern der Ergebnisse}
	\begin{itemize}
	\item in einem ein- oder mehrdimensionalen Array 
	\item Basisfälle vorinitialisieren 
	\item Rest der Tabelle wird mit Werten für 'nicht berechnet' zB. -1 initialisiert
	\item für jeden Funktionsaufruf Table-Lookup:
		\begin{itemize}
			\item steht ein Wert in der Tabelle: zurückgeben
			\item sonst: Wert berechnen, in der Tabelle speichern und zurückgeben
		\end{itemize}
	\end{itemize}
\end{frame}

\begin{frame}{Top-down}
	\begin{itemize}
		\item 'Vom Problem zur Lösung'
		\item rekursiver Funktionsaufruf
		\item overhead durch viele Funktionsaufrufe
		\item es kann bei extrem großen Problemen zu stackoverflow kommen
	\end{itemize}
\end{frame}

\begin{frame}{Besser: Bottom-Up}
	\begin{itemize}
		\item die Tabelle wird sukzessiv mit Lösungen gefüllt
		\item das Endergebniss wird danach einfach aus der Tabelle abgelesen
	\end{itemize}
\end{frame}

\begin{frame}{Levenshtein-Distance}
	Die sog. \emph{Levenshtein Distanz} gibt eine Editierdistanz, also eine Art Abstand, zwischen zwei Wörtern an.
	Sie ist definiert als die minimale Anzahl an Einfüge-, Lösch- und Ersetzungsoperationen, um ein Wort in das andere umzuwandeln.

	Lösung mit dynamischer Programmierung!

	$$
	lev_{a,b}(i, j) = min \begin{cases}
		lev_{a,b}(i - 1, j) + 1\ \text{\emph{(a einfügen/b löschen)}}\\
		lev_{a,b}(i, j - 1) + 1\ \text{\emph{(a löschen/b einfügen)}}\\
		lev_{a,b}(i - 1, j - 1) + 1_{(a_i \neq b_j)}\ \text{\emph{(ersetzen)}}\\
	\end{cases}
	$$
\end{frame}

\begin{frame}{Palindrome}
	Palindrome sind Wörter, die rückwärts gleich wie vorwärts sind (z.B. "`anna"', "`kajak"', "`reittier"').

	\textbf{Aufgabe 1}: Entwickle einen Algorithmus, der für eine Zeichenkette der Länge $n$ in $\mathcal{O}(n^2)$ Zeit die Länge der längsten palindromischen Teilsequenz.

	Beispiel: "`FA\textbf{K}UL\textbf{T}Ä\textbf{TSF}E\textbf{ST} INFORMA\textbf{T}I\textbf{K} 2014"'

	\textbf{Aufgabe 2}: Erweitere den Algorithmus, sodass er zusätzlich diese Sequenz ausgibt.
\end{frame}

\begin{frame}{Miners}
	Zwei Kohleminen müssen mit Essen versorgt werden.
	Es gibt drei verschiedene Arten von Essen: Fleisch, Fisch und Brot.
	Die Arbeiter mögen Abwechslung und sind produktiver, wenn sie verschiedene Arten von Essen bekommen.
	Für die nächste Produktion entscheidend sind immer die letzten drei Sendungen.

	\begin{itemize}
		\item Wenn alle Sendungen gleich sind $\Rightarrow$ 1 Kohleeinheit
		\item Wenn es zwei verschiedene Sendungen gab $\Rightarrow$ 2 Kohleeinheiten
		\item Wenn alle verschieden waren $\Rightarrow$ 3 Kohleeinheiten
	\end{itemize}

	Für eine gegebene Reihenfolge an Essenssendungen kann jeweils entschieden werden, welche Mine sie erhält.

	\textbf{Aufgabe}: Berechne die maximale Anzahl an produzierten Kohleeinheiten.
\end{frame}

\end{document}
