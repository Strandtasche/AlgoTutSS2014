\input{includes/head}
\title[Algorithmen I SS 14]{Tutorium 9}

\usepackage{alltt}

\TitleImage[width=\titleimagewd]{images/path}

\begin{document}

\begin{frame}
  \maketitle
\end{frame}

\begin{frame}{Minimale Spannbäume}
	\begin{description}
		\item[Gegeben:] Ungerichteter Graph mit Knotenmenge $V$, Kantenmenge $E$ und einer Gewichtsfunktion c
		\item[Gesucht:] Baum mit Knotenmenge $V$, Kantenmenge $F$, $F \subseteq E$ mit minimaler $\sum_{{
   e \in T }} c(e)$, der alle Knoten verbindet 
	\end{description}

	\includegraphics[scale = 0.25]{images/graphs}
\end{frame}

\begin{frame}{Schnitteigenschaft}
	Für eine Beliebige Knotenmenge $S \subseteq V$ gilt für die Kantenmenge C: \\
	\ \\
	\centerline{$C = \{ \{u,v\} \in E: u \in S, v \in V \  \textbackslash \  S \}$}
	\ \\
	\ \\
	$\Rightarrow$ Die Kante mit dem geringsten Gewicht aus $C$ ist Teil des MST
\end{frame}

\begin{frame}{Beispiel}
	\includegraphics[width=\textwidth]{images/mst}
\end{frame}

\end{document}
