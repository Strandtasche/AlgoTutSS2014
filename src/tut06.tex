\input{includes/head}
\title[Algorithmen I SS 14]{Tutorium 4}

\usepackage{alltt}

\TitleImage[height=\titleimageht]{images/sortinghat}

\definecolor{english}{rgb}{0.0, 0.5, 0.0}

\begin{document}

\begin{frame}
  \maketitle
\end{frame}

\begin{frame}{Heaps}
	Eine baumartige Datenstruktur:
	\begin{itemize}
		\item Die Wurzel jedes Subtrees ist in der Ordnungsrelation größer als alle Elemente unter ihm.
		\item Häufige Ordnungen auf Heaps:
		\begin{itemize}
			\item kleiner-gleich: min-Heap
			\item größer-gleich: max-Heap
		\end{itemize}
	\end{itemize}
	\ \\
	\ \\
	\centerline{Im weiteren betrachten wir nur \textbf{Binäre Heaps}}

\end{frame}

\begin{frame} {Heap Operationen}
	\begin{itemize}
		\item Wurzel betrachten in $\mathcal{O}(1)$
		\begin{itemize}
			\item in min-Heaps: das Minimum
			\item in max-Heaps: das Maximum
		\end{itemize}
		\item Heap Eigenschaft herstellen in $\mathcal{O}(n)$
		\item Heap reparieren in $\mathcal{O}(\log  n)$
	\end{itemize}
	Daraus folgt:
	\begin{itemize}
		\item Einfügen in $\mathcal{O}($log $n )$
		\item Wurzel extrahieren in $\mathcal{O}( \log \ n)$
	\end{itemize}

\end{frame}

\begin{frame}{Implementierung}
	Im Computer effizient als Array darstellbar: \\
	\includegraphics[scale=0.2]{images/heapArray.png} \\
	$\Rightarrow$ Traversierung des Baums?
\end{frame}

\begin{frame}{Baumnavigation in Array Darstellung}
	\begin{description}
		\item[leftChild(i):] heap[$2*i$]
		\item[rightChild(i):] heap[$2*i + 1$]
		\item[parent(i):] heap[$\lfloor i / 2 \rfloor$]
	\end{description}
	\ \\
	\ \\
	Ein Heap der Höhe h hat mindestens $2^{h - 1} $ und maximal $ 2^{h} - 1$ Elemente \\
	\ \\
	Ein Heap mit n Elementen hat die Höhe $\lceil \log_2(n) \rceil$
\end{frame}



\end{document}
