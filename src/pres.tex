\input{includes/head}
\title[Algorithmen I SS 14]{Tutorium 4}

\usepackage{alltt}

\TitleImage[height=\titleimageht]{images/sortinghat}

\definecolor{english}{rgb}{0.0, 0.5, 0.0}

\begin{document}

\begin{frame}
  \maketitle
\end{frame}

\begin{frame}{Eigenschaften von Sortieralgorithmen}
	\begin{block}{in-place}
		Benötigt nur konstant viel zusätzlichen Speicherplatz.
	\end{block}
	\begin{block}{stabil}
		Gleiche Elemente werden nicht vertauscht.

		$\langle 3, {\color{red}{2}}, 1, {\color{english}{2}}\rangle$ (unsortiert)\\
		$\langle 1, {\color{english}{2}}, {\color{red}{2}}, 3\rangle$ (nicht-stabil sortiert)\\
		$\langle 1, {\color{red}{2}}, {\color{english}{2}}, 3\rangle$ (stabil sortiert)
	\end{block}
\end{frame}

\begin{frame}
	\frametitle{Quicksort}
	Rekursiver Sortieralgorithmus
	\begin{itemize}
		\item Funktionsweise:
		\begin{itemize}
			\item Divide-and-Conquer:
			\item Aufspaltung mit Pivot Element
			\item Rekusiver Aufruf auf die Überreste der Aufspaltung
		\end{itemize}
		\item Laufzeiten:
		\begin{itemize}
			\item Best Case: $\mathcal{O}(n \log{n})$
			\item Average Case: $\mathcal{O}(n \log{n})$
			\item Worst Case: $\mathcal{O}(n^2)$
		\end{itemize}
		\item {\color{orange}  "`inplace"'} 
		\item {\color{red} nicht stable}
	\end{itemize}
\end{frame}

\begin{frame}{Beispiel}
	Sortiere $\langle 58, 38, 97, 68, 6, 21, 37, 54, 24, 16, 65\rangle$

	\begin{enumerate}
		\item Quicksort (Pivot: erstes Element)
	\end{enumerate}
\end{frame}


\begin{frame}
	\frametitle{Ineffective Sorts}
	\begin{center}
		\includegraphics[width=\textwidth,height=\textheight,keepaspectratio]{images/ineffective_sorts}
	\end{center}
\end{frame}

\end{document}
