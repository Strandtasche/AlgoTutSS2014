%% LaTeX-Beamer template for KIT design
%% by Erik Burger, Christian Hammer
%% title picture by Klaus Krogmann
%%
%% version 2.1
%%
%% mostly compatible to KIT corporate design v2.0
%% http://intranet.kit.edu/gestaltungsrichtlinien.php
%%
%% Problems, bugs and comments to
%% burger@kit.edu

\documentclass[18pt]{beamer}
\usepackage[utf8]{inputenc}
\usepackage{algpseudocode}
\usepackage{listings}
\usepackage{amsfonts} 
\usepackage{mathtools} 
%% SLIDE FORMAT

% use 'beamerthemekit' for standard 4:3 ratio
% for widescreen slides (16:9), use 'beamerthemekitwide'

\usepackage{templates/beamerthemekit}
% \usepackage{templates/beamerthemekitwide}

%% TITLE PICTURE

% if a custom picture is to be used on the title page, copy it into the 'logos'
% directory, in the line below, replace 'mypicture' with the 
% filename (without extension) and uncomment the following line
% (picture proportions: 63 : 20 for standard, 169 : 40 for wide
% *.eps format if you use latex+dvips+ps2pdf, 
% *.jpg/*.png/*.pdf if you use pdflatex)

%\TitleImage[width=\titleimagewd]{pics/title}
%\titleimage{}

%% TITLE LOGO

% for a custom logo on the front page, copy your file into the 'logos'
% directory, insert the filename in the line below and uncomment it

%\titlelogo{mylogo}

% (*.eps format if you use latex+dvips+ps2pdf,
% *.jpg/*.png/*.pdf if you use pdflatex)

%% TikZ INTEGRATION

% use these packages for PCM symbols and UML classes
% \usepackage{templates/tikzkit}
% \usepackage{templates/tikzuml}
\usepackage{listings}
\usepackage{color}

\definecolor{dkgreen}{rgb}{0,0.6,0}
\definecolor{gray}{rgb}{0.5,0.5,0.5}
\definecolor{mauve}{rgb}{0.58,0,0.82}

\lstset{frame=tb,
  language=Java,
  aboveskip=3mm,
  belowskip=3mm,
  showstringspaces=false,
  columns=flexible,
  basicstyle={\small\ttfamily},
  numbers=none,
  numberstyle=\tiny\color{gray},
  keywordstyle=\color{blue},
  commentstyle=\color{dkgreen},
  stringstyle=\color{mauve},
  breaklines=true,
  breakatwhitespace=true
  tabsize=3
}


\setbeamertemplate{enumerate items}[circle]
% the presentation starts here

\title[Algo Tutorium Nr.2]{Algorithmen I - Sommersemester 2014}
\subtitle{Tutorium Nr. 2}
\author{Tobias Hornberger}

\institute{Institut für Theoretische Informatik}

\begin{document}

% change the following line to "ngerman" for German style date and logos
\selectlanguage{ngerman}

%title page
\begin{frame}
\titlepage
\end{frame}

%table of contents
%~ \begin{frame}[allowframebreaks]{Übersicht}
%~ \tableofcontents
%~ \end{frame}

\section{Organisatorisches}
\subsection{Übungsblätter}
	\begin{frame}
		\begin{center}
			\huge{Rückgabe der Übungsblätter} \\
			\parskip 36pt
			\normalsize{$\rightarrow$ Bei Rückfragen bitte am Ende des Tutoriums vorkommen}
		\end{center}
	\end{frame}

\subsection{Email-Liste}

	\begin{frame}{Email-Liste}
		Nach dem Email-Desaster: \\
		\parskip 20pt
		\begin{block}{Kommunikation}
			\begin{itemize}
				\item Bitte Email auf dem Blatt eintragen
				\item Nach dem Tutorium schauen ob meine Email angekommen ist
			\end{itemize}
		\end{block}
	\end{frame}

\section{Rekurrenzen}
\subsection{Erklärung}
	\begin{frame}{Rekurrenz Beispiel}
		\begin{center}
			$ T(n)=\left\{\begin{array}{cr} 1, & \mbox{falls } n = 1\\ 2 T(\lfloor n / 2 \rfloor) + n, & \mbox{falls } n \geq 2 \end{array}\right. $

			\parskip 14pt
			$n \in 2 ^{m} $ für $ m \in \mathbb{N} _{0} $

		\end{center}
		\begin{block}{Vorgehen}
			\begin{enumerate}
				\item Lösung raten...
				\item Beweisen, dass die Lösung stimmt
			\end{enumerate}
		\end{block}

	\end{frame}
	
	\begin{frame}
		Umformung: \\
		$T(2^m) = 2* T(2^{m - 1}) + 2^m$ mit $T(1) = 1$\\
		$S(m) = 2 * S(m) + 2^m$ mit $S(0) = 1$ \\
		\parskip 12pt
		Ersten Werte:
		\begin{itemize}
			\item $S(0) = T(2^0) = T(1) = 1$
			\item $S(1) = T(2^1) = T(2) = 2 * 1 + 2 = 4$
			\item $S(2) = T(2^2) = T(4) = 2 * (2 * 1 + 2) + 4 = 12$
			\item $S(3) = T(2^3) = T(8) = 2 * (2 * (2 * 1 + 2) + 4 ) + 8 = 32$
			\item $S(4) = T(2^4) = T(16) = 2 * ( 2 * (2 * (2 * 1 + 2) + 4 ) + 8 ) + 16 = 70$
		\end{itemize}
	\end{frame}

	\begin{frame}{Rekurrenzen}
		Durch scharfes Hinsehen erkennen wir eine mögliche Lösung: \\
		\ \\
		\centerline{\large{$S(m) = 2^{m+1} + (m-1) * 2^{m}$}}
		\ \\
		Beweis der Lösung durch völlständige Induktion.
	\end{frame}

	\begin{frame}{Beweis durch vollständige Induktion}
		\textbf{IA:} $S(0) = T(2^0) = T(1) = 1$ \\
		\textbf{IV:} $S(m) = 2^{m+1} + (m - 1)*2^{m}$\\
		\textbf{IS:} m $\rightarrow$ $m+1$
		\begin{align}
			S(m+1)& = 2 * S(m) + 2^{m+1} \\
			& = 2 * ( 2^{m+1} + (m-1) * 2^m) + 2^{m+1} \\
			& = 2^{m+2} + (m-1) * 2^{m+1} + 2^{m+1} \\
			& = 2^{m+2} + m * 2^{m+1}
		\end{align}
		\ \\
		Zurück nach T(n) rechnen für Finale Lösung: \\
		$T(n) = S(\log _2 n) = 2^{\log _2 n + 1} * (\log _2 n - 1) * 2 ^{\log _2 n}$
	\end{frame}


\section{Datenstrukturen}
	\begin{frame}{Test Test}
		Test
	\end{frame}


\end{document}
