 \documentclass[c]{beamer}
%\documentclass{beamer}
\listfiles

\mode<presentation>
{
  %\usetheme[deutsch,titlepage0]{KIT}
\usetheme[deutsch]{KIT}
% \usetheme{KIT}

%%  \usefonttheme{structurebold}

  \setbeamercovered{transparent}

  \setbeamertemplate{enumerate items}[circle]
  %\setbeamertemplate{enumerate items}[ball]

}
\usepackage{babel}
\date{}
%\DateText

\newlength{\Ku}
\setlength{\Ku}{1.43375pt}

\usepackage[utf8]{inputenc}
\usepackage[TS1,T1]{fontenc}
\usepackage{array}
\usepackage{multicol}
\usepackage{lipsum}
\usepackage[]{algorithm2e}
\usepackage{amsmath}
\usepackage{color}

\usenavigationsymbols
%\usenavigationsymbols[sfHhdb]
%\usenavigationsymbols[sfhHb]

\subtitle{Algorithmen I SS 14}
\author[]{Lena Winter}

\AuthorTitleSep{\relax}

\institute[ITI]{Institut für Theoretische Informatik}

\TitleImage[width=\titleimagewd]{images/title}

\newlength{\tmplen}

\newcommand{\verysmall}{\fontsize{6pt}{8.6pt}\selectfont}

\title[Algorithmen I SS 14]{Tutorium 2}

\begin{document}

\begin{frame}
  \maketitle
\end{frame}

\begin{frame}
	\frametitle{Datenstrukturen: Überblick}
	\textbf{Listen:}\\
	\begin{itemize}
		\item \textit{List:} doppelt verkette Liste
		\item \textit{SList:} einfach verkette Liste
	\end{itemize}

	\textbf{Arrays:}\\
	\begin{itemize}
		\item \textit{UArray:} unbounded Array
		\item \textit{CArray:} cyclic Array
	\end{itemize}
\end{frame}


\begin{frame}
	\frametitle{amortisierte Analyse}
	\begin{itemize}
		\item gemittelter Aufwand für eine Sequenz von Operationen im Worst-case
		\item Kosten einer Operation gering, auch wenn eine einzelne Operation kostspielig ist
		\item keine Einbeziehung von Wahrscheinlichkeiten
		\item Beispiel: unbounded Arrays 
	\end{itemize}
\end{frame}

\begin{frame}
	\frametitle{Konto-Methode}
	für jede Operation:
	\begin{itemize}
		\item $\hat{c}_i$ \textbf{amortisierte Kosten}
		\item $c_i$ \textbf{tatsächliche Kosten}
		\item $\hat{c}_i \geq c_i  \Rightarrow $  Guthaben einzahlen
		\item $\hat{c}_i \leq c_i  \Rightarrow $  Guthaben abheben
		\item Konto darf \textbf{nie} negativ werden
		\begin{itemize}
			\item $ \forall n \in \mathbb{N} : \sum_{i=1} ^n \hat{c}_i \geq \sum_{i=1} ^n c_i$
		\end{itemize}

	\end{itemize}

\end{frame}

\begin{frame}
	\frametitle{Beispiel: Unbounded Arrays}

	\begin{itemize}
		\item amortisierte Kosten:
		\begin{itemize}
			\item \textbf{push}: 3 Tokens
			\begin{itemize}
				\item 1 Token: Einfügen des Elements
				\item 2 Tokens: Konto
			\end{itemize}
			\item \textbf{pop}: 2 Tokens
			\begin{itemize}
				\item 1 Token: Löschen des Elements
				\item 1 Token: Konto
			\end{itemize}
		\end{itemize}
		\item falls ein Feld mit \textit{n} Elementen durch \textbf{reallocate} verkleinert oder vergrößert wird,
		 müssen \textit{n} Elemente kopiert werden $\Rightarrow$ \textit{n} Tokens müssen vorhanden sein

	\end{itemize}

\end{frame}

\begin{frame}
	\frametitle{Beispiel: Unbounded Arrays}
	\begin{itemize}
		\item Anfangszustand nach \textbf{reallocate}:
		\begin{itemize}
			\item \textit{n} Elemente, kein Guthaben vorhanden \\ \ \\
		\end{itemize}


		\item Sequenz von \textbf{push}-Operartionen:
		\begin{itemize}
			\item \textbf{reallocate} wird nach \textit{n} \textbf{push}-Operartionen aufgerufen
			\item Guthaben: 2\textit{n} Tokens
			\item \textbf{reallocate}-Aufruf: 2\textit{n} Elemente werden kopiert $\rightarrow$ genug Tokens \\ \ \\
		\end{itemize} 

		\item Sequenz von \textbf{pop}-Operationen:
		\begin{itemize}
			\item \textbf{reallocate} wird nach \textit{$\frac{n}{2}$} \textbf{push}-Operartionen aufgerufen
			\item Guthaben: \textit{$\frac{n}{2}$} Tokens
			\item \textbf{reallocate}-Aufruf: \textit{$\frac{n}{2}$} Elemente werden kopiert $\rightarrow$ genug Tokens \\ \ \\
		\end{itemize} 

	\end{itemize}
\end{frame}


\begin{frame}
	\frametitle{Kreativaufgabe}
	Entwickeln Sie eine Datenstruktur die folgendes kann:
	\begin{itemize}
		\item \textit{pushBack} und \textit{popBack} in  $\mathcal{O}(1)$ im Worst-Case nicht nur amortisiert.
		\item Zugriff auf das k-te Element in $\mathcal{O}(\log n)$ im Worst-Case nicht nur amortisiert.
	\end{itemize}
	Wobei ein eine Speicherallokation beliebiger Größe nur O(1) kosten soll.\\
	\ \\
	\textbf{Zusatzaufgabe:}\\
	\begin{itemize}
		\item \textit{pushBack} und \textit{popBack} in  $\mathcal{O}(\log n)$ im Worst-Case nicht nur amortisiert.
		\item Zugriff auf das k-te Element in $\mathcal{O}(1)$ im Worst-Case nicht nur amortisiert.
	\end{itemize}

\end{frame}


\begin{frame}
	\frametitle{Rekurrenzen}

\end{frame}

\end{document}
\end{document}
