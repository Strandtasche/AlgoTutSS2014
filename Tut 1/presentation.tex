%% LaTeX-Beamer template for KIT design
%% by Erik Burger, Christian Hammer
%% title picture by Klaus Krogmann
%%
%% version 2.1
%%
%% mostly compatible to KIT corporate design v2.0
%% http://intranet.kit.edu/gestaltungsrichtlinien.php
%%
%% Problems, bugs and comments to
%% burger@kit.edu

\documentclass[18pt]{beamer}
\usepackage[utf8]{inputenc}
%% SLIDE FORMAT

% use 'beamerthemekit' for standard 4:3 ratio
% for widescreen slides (16:9), use 'beamerthemekitwide'

\usepackage{templates/beamerthemekit}
% \usepackage{templates/beamerthemekitwide}

%% TITLE PICTURE

% if a custom picture is to be used on the title page, copy it into the 'logos'
% directory, in the line below, replace 'mypicture' with the 
% filename (without extension) and uncomment the following line
% (picture proportions: 63 : 20 for standard, 169 : 40 for wide
% *.eps format if you use latex+dvips+ps2pdf, 
% *.jpg/*.png/*.pdf if you use pdflatex)

%\titleimage{mypicture}

%% TITLE LOGO

% for a custom logo on the front page, copy your file into the 'logos'
% directory, insert the filename in the line below and uncomment it

%\titlelogo{mylogo}

% (*.eps format if you use latex+dvips+ps2pdf,
% *.jpg/*.png/*.pdf if you use pdflatex)

%% TikZ INTEGRATION

% use these packages for PCM symbols and UML classes
% \usepackage{templates/tikzkit}
% \usepackage{templates/tikzuml}

% the presentation starts here

\title[Algo Tutorium Nr.1]{Algorithmen I - Sommersemester 2014}
\subtitle{Tutorium Nr. 1}
\author{Tobias Hornberger}

\institute{Institut für Theoretische Informatik}

\begin{document}

% change the following line to "ngerman" for German style date and logos
\selectlanguage{ngerman}

%title page
\begin{frame}
\titlepage
\end{frame}

%table of contents
%~ \begin{frame}[allowframebreaks]{Übersicht}
%~ \tableofcontents
%~ \end{frame}

\section{Einführung}
	\subsection{Vorstellung}
	\begin{frame}{Der Tutor}
		Tobias Hornberger
		\begin{itemize}
			\item 4. Semester Informatik
			\item Algorithmen I vor 2 Semestern gehört
			\item erstes Mal Tutor
		\end{itemize}
		Vorschläge/Anmerkungen/Feedback ist also sehr erwünscht
	\end{frame}

	\subsection{Kontakt}
	\begin{frame}{Kontakt}
		
		\begin{itemize}
			\item Email: \texttt{saibot1207@googlemail.com}
			\item Github: *TODO*
			\item Bei Fragen: Antwort an alle
		\end{itemize}
	\end{frame}
	
	\begin{frame}{Namen}
		Vorstellungsrunde		
	\end{frame}

	\subsection{Organisatorisches}

	\begin{frame}{Organisation und Übungsbetrieb}
		\begin{itemize}
			\item Das Tutorium ist nicht dazu gedacht Übung/Vorlesung zu ersetzen, sondern sie zu ergänzen.
			\item Der Besuch wird empfohlen
			\item Keine Frontalveranstaltung
			\begin{itemize}
				\item Wenn Fragen auftauchen: einfach Fragen
				\item Es gibt keine dummen Fragen
			\end{itemize}
		\end{itemize}
	\end{frame}

	\begin{frame}{Organisation und Übungsbetrieb}
		Speziell für dieses Semester:
		\begin{itemize}
			\item Viele Feiertage fallen auf Donnerstage
			\item in diesen Wochen bitte andere Tutorien besuchen 
		\end{itemize}		
	\end{frame}

	\begin{frame}{Homepages}
		\begin{itemize}
			\item Vorlesungshomepage: \texttt{http://algo2.iti.kit.edu/AlgorithmenI2014.php}
			\begin{itemize}
				\item Interessante Organisatorische Details
				\item Vorlesungsfolien
				\item Die Übungsblätter
				\item Prof. Sanders Buch
				\item Literaturliste
			\end{itemize}
			\item Meine Folien: \texttt{https://github.com/Strandtasche/AlgoTutSS2014}
			\item ILIAS: \texttt{https://ilias.studium.kit.edu}
			\item Fakultät für Informatik $\rightarrow$ SS 2014 $\rightarrow$ Algorithmen I mit Übung			
		\end{itemize}		
	\end{frame}

	\begin{frame}{Übungsblätter}
		Das meiste wurde schon in der vorlesung gesagt:
		\begin{itemize}
			\item 9 Tage Bearbeitungszeit: von Mittwoch bis Freitag der nächsten Woche
			\item Abgabe 12:45 Uhr im Kasten hier vor dem Raum (-118)
			\item 2er-Teams sind erlaubt
			\item Teamwechsel im Semester sind nicht vorgesehen, können aber in Sonderfälle mit Rücksprache durchgeführt werden
			\item Wichtig Wichtig: Tutoriumsnummer rechts oben groß auf die Abgabe!
			\item Kein Verpflichtender Übungsschein sondern Klausurbonus von 1/2/3 Punkten
			\item Zum ersten Mal dieses Jahr: Programmieraufgaben
			\begin{itemize}
				\item 4 Stück, über das Semester verteilt, in Java
				\item keine Klassenstrukturen sondern tatsächlich nur der Algorithmus
				\item Praktomat Abgabe: Ohne Verpflichtenden Code Style
				\item Dazu mehr wenn die erste Aufgabe kommt...
			\end{itemize}
		\end{itemize}		
	\end{frame}

	\begin{frame}{Beispiel}
		\begin{figure}
				$(\lambda x.x x)(\lambda y.y) z$ 
				$\;\Leftrightarrow\;$
				$\vcenter{\hbox{\includegraphics[width=4cm,type=pdf,ext=.pdf,read=.pdf]{logos/(Ax.xx)(Ay.y)z}}}$
		\end{figure}
		\vspace{-1cm}
		\pause
		\begin{figure}
				$(\lambda y.y)(\lambda y.y) z$ 
				$\;\Leftrightarrow\;$
				\visible<2->{$\vcenter{\hbox{\includegraphics[width=4cm,type=pdf,ext=.pdf,read=.pdf]{logos/(Ay.y)(Ay.y)z}}}$}
		\end{figure}
		\vspace{-1cm}
		\pause
		\begin{figure}
				$(\lambda y.y) z$ 
				$\;\Leftrightarrow\;$
				\visible<3->{$\vcenter{\hbox{\includegraphics[width=3.5cm,type=pdf,ext=.pdf,read=.pdf]{logos/(Ay.y)z}}}$}
		\end{figure}
		\vspace{-1cm}
		\pause
		\begin{figure}
				$z$ 
				$\;\Leftrightarrow\;$
				\visible<4>{$\vcenter{\hbox{\includegraphics[width=2cm,type=pdf,ext=.pdf,read=.pdf]{logos/z}}}$}
		\end{figure}
		\vspace{-1cm}
		\vfill
		\pause
		$\rightarrow$ Kindgerecht, aber ohne Zielstellung langweilig
	\end{frame}

\section{Produktvorstellung}
	\subsection{Croggle, die App}
	\begin{frame}{Croggle in Aktion}
		\center{Wie löst Croggle dieses Problem?}
	\end{frame}
	
	\subsection{Leveltypen}
	\begin{frame}[<+->]{Spielmechanik in Croggle}
		Drei verschiedene Spielmodi:
		\begin{itemize}
			\item Multiple Choice
			\item Vorgegebene Lücken füllen
			\item Terme ergänzen
		\end{itemize}
	\end{frame}

	\subsection{Zusatzfunktionen}
	\begin{frame}[<+->]{Weitere Features}
		\begin{itemize}
			\item Mehrere Benutzer
			\item Statistiken
			\item Achievements
			\item Farbenblindenmodus
			\item Mehrsprachigkeit
		\end{itemize}
		und vieles mehr!
	\end{frame}

\section{Umsetzung}
	\begin{frame}{Realisierung}
		\center{Wie haben wir das geschafft?}
	\end{frame}

	\subsection{Planung \& Entwurf}
	\begin{frame}[<+->]{Tools}
		\begin{block}{Planung}
			\begin{itemize}
				\item LaTeX für alle Dokumente
				\item Gimp für erste Screen Mockups
				\item ArgoUML für das UML der Planungsphase
			\end{itemize}
		\end{block}
		\begin{block}{Entwurf}
			\begin{itemize}
				\item UMLet für Sequenzdiagramme der Entwurfsphase
				\item Nichts drückt den Inhalt eines Softwareprojekts besser aus als sein Quellcode \\
				$\rightarrow$ Java Quellcode bereits für den Entwurf
				\item KIT hauseigenes TexDoclet zur Generierung der Klassendokumentation aus Java Code
				\item UMLGraph zur Erstellung von Klassendiagrammen aus Java Quellcode
			\end{itemize}
		\end{block}
	\end{frame}

	\subsection{Implementierung \& Test}
	\begin{frame}[<+->]{Tools}
		\begin{block}{Implementierung}
			\begin{itemize}
				\item Eclipse IDE
				\item Android SDK rev19
				\item libGdx Framework
				%~ \item Nuzung von libGdx' TexturePacker
				\item git zur Revisionskontrolle
				\item GitHub: \textcolor{blue}{\href{http://github.com/TeamCroggle}{github.com/TeamCroggle}} \\
					$\rightarrow$ Open Source
			\end{itemize}
		\end{block}
		\begin{block}{Testphase}
			\begin{itemize}
				\item jUnit 3 (Version 4 von Android nicht unterstützt)
				\item Emma, einschließlich Eclipse Integration EclEmma
				\item MonkeyRunner und Monkey Tool aus dem Android SDK
			\end{itemize}
		\end{block}
	\end{frame}
	
	\begin{frame}[<+->]{Nebenprodukt Desktopversion}
		\begin{itemize}
			\item Bereits in der Demo gezeigt
			\item Enstanden zwischen Implementierung und Test
			\item Durch Framework einfach gemacht
			\item Teilweise schon während Implementierung auf Portabilität geachtet
			\item Hauptarbeit SQLite
		\end{itemize}
	\end{frame}

	\subsection{Problemstellungen}
	\begin{frame}[<+->]{Herausforderungen}
		Die größten Hürden, die es zu bewältigen galt: 
		\begin{itemize}
			\item Einarbeitung ins $\lambda$-Kalkül
			\item Einarbeitung Android bzw. libGdx
			\item Darstellung der Alligatoren getrennt nach MVC
			\item Eigenes auf JSON aufbauendes Format für Level und Tutorials
			\item Speicherung der Nutzerdaten in Sqlite
			\item Und natürlich der ganze Rest der App
			%~ \item Eigene Abstraktion der Android Lokalisierungsdienste für Plattformunabhängigkeit
		\end{itemize}
	\end{frame}

	%~ \begin{frame}{Probleme}
		%~ \begin{itemize}
			%~ \item Verschiedene Entwicklungsumbebungen und Softwareversionen
			%~ \pause
			%~ \item Bug in libGdx $\rightarrow$ Bug Report
			%~ \pause
			%~ \item Keine WYSIWYG Editoren für die GUI
		%~ \end{itemize}
	%~ \end{frame}

\section{Rezeption \& Fazit}
	\subsection{Meinungen von Außerhalb}
	\begin{frame}[<+->]{Meinungen von Außerhalb}
		Das meinen die Tester:
		\begin{itemize}
			\item Tolle Grafiken
			\item Hier und da müssten zusätzliche Erklärungen hin
			\item Wirkt sehr professionell
			\item Analogie Alligator $\Leftrightarrow$ Lambdakalkül hinkt/verwirrt
		\end{itemize}
	\end{frame}

	\subsection{Eigene Meinung}
	\begin{frame}[<+->]{Unsere Meinung}
		Das meinen wir:
		\begin{itemize}
			\item Tolles Projekt: Viele Freiheiten ohne aber überhaupt erst mal ein Konzept finden zu müssen
			\item Sehr viel gelernt, sowohl technisch wie auch bzgl. Teamarbeit
			\item Mehr Wagemut zum Abändern der Spielidee, damit es logischer wird, wäre gut gewesen
			\item Gute Balance gefunden zwischen Menge an Features und insgesamter Ausgereiftheit
			\item Ein bisschen zu viel vorgenommen $\rightarrow$ wenig Freizeit
		\end{itemize}
	\end{frame}

	%~ \subsection{Lehren aus dem Projekt}
	%~ \begin{frame}{Was wir gelernt haben}
	%~ \end{frame}

\section{Ende}
	\begin{frame}{Ende}
		\center{Vielen Dank für Eure Aufmerksamkeit}
	\end{frame}

	%~ \subsection{Amüsantes}
	%~ \begin{frame}{Bloopers}
		%~ \begin{itemize}
			%~ \item Für den Entwurf wurde das KIT TexDoclet von uns geforkt und so stark modifiziert, dass mir vorgeschlagen wurde, Maintainer zu werden
			%~ \item Inhaltsverzeichnis ohne latexmk kaputt
			%~ \item Unübersichtliches Klassendiagramm dank Autogenerierung
			%~ \item Umstellung Repository kurz vor der Testphase
		%~ \end{itemize}
	%~ \end{frame}
%~ 
	%~ \begin{frame}{Miscellaneous}
	%~ Für weitere Stichpunkte
		%~ \begin{itemize}
			%~ \item Statistiken und genaue Zahlen über weiterverwendbaren Code
			%~ \item Statistiken hier einfügen
			%~ \item Zielsetzung: Altersgruppe, Geräteanforderungen, Langzeitmotivation...
		%~ \end{itemize}
	%~ \end{frame}

\end{document}
