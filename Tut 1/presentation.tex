%% LaTeX-Beamer template for KIT design
%% by Erik Burger, Christian Hammer
%% title picture by Klaus Krogmann
%%
%% version 2.1
%%
%% mostly compatible to KIT corporate design v2.0
%% http://intranet.kit.edu/gestaltungsrichtlinien.php
%%
%% Problems, bugs and comments to
%% burger@kit.edu

\documentclass[18pt]{beamer}
\usepackage[utf8]{inputenc}
\usepackage{algpseudocode}
%% SLIDE FORMAT

% use 'beamerthemekit' for standard 4:3 ratio
% for widescreen slides (16:9), use 'beamerthemekitwide'

\usepackage{templates/beamerthemekit}
% \usepackage{templates/beamerthemekitwide}

%% TITLE PICTURE

% if a custom picture is to be used on the title page, copy it into the 'logos'
% directory, in the line below, replace 'mypicture' with the 
% filename (without extension) and uncomment the following line
% (picture proportions: 63 : 20 for standard, 169 : 40 for wide
% *.eps format if you use latex+dvips+ps2pdf, 
% *.jpg/*.png/*.pdf if you use pdflatex)

%\titleimage{mypicture}

%% TITLE LOGO

% for a custom logo on the front page, copy your file into the 'logos'
% directory, insert the filename in the line below and uncomment it

%\titlelogo{mylogo}

% (*.eps format if you use latex+dvips+ps2pdf,
% *.jpg/*.png/*.pdf if you use pdflatex)

%% TikZ INTEGRATION

% use these packages for PCM symbols and UML classes
% \usepackage{templates/tikzkit}
% \usepackage{templates/tikzuml}

% the presentation starts here

\title[Algo Tutorium Nr.1]{Algorithmen I - Sommersemester 2014}
\subtitle{Tutorium Nr. 1}
\author{Tobias Hornberger}

\institute{Institut für Theoretische Informatik}

\begin{document}

% change the following line to "ngerman" for German style date and logos
\selectlanguage{ngerman}

%title page
\begin{frame}
\titlepage
\end{frame}

%table of contents
%~ \begin{frame}[allowframebreaks]{Übersicht}
%~ \tableofcontents
%~ \end{frame}

\section{Einführung}
	\subsection{Vorstellung}
	\begin{frame}{Der Tutor}
		Tobias Hornberger
		\begin{itemize}
			\item 4. Semester Informatik
			\item Algorithmen I vor 2 Semestern gehört
			\item erstes Mal Tutor
		\end{itemize}
		Vorschläge/Anmerkungen/Feedback ist also sehr erwünscht
	\end{frame}

	\subsection{Kontakt}
	\begin{frame}{Kontakt}
		
		\begin{itemize}
			\item Email: \texttt{saibot1207@googlemail.com}
			\item Github: *TODO*
			\item Bei Fragen per Email: Antwort an Alle
			Email mit Betreff "Algo1Tut" an mich schicken.
		\end{itemize}
	\end{frame}
	
	\begin{frame}{Namen}
		Vorstellungsrunde		
	\end{frame}

	\subsection{Organisatorisches}

	\begin{frame}{Organisation und Übungsbetrieb}
		\begin{itemize}
			\item Das Tutorium ist nicht dazu gedacht Übung/Vorlesung zu ersetzen, sondern als Ergänzung.
			\item Der Besuch wird empfohlen
			\item Keine Frontalveranstaltung
			\begin{itemize}
				\item Wenn Fragen auftauchen: einfach Fragen
				\item Es gibt keine dummen Fragen
			\end{itemize}
		\end{itemize}
	\end{frame}

	\begin{frame}{Organisation und Übungsbetrieb}
		Speziell für dieses Semester:
		\begin{itemize}
			\item Viele Feiertage fallen auf Donnerstage
			\item in diesen Wochen bitte andere Tutorien besuchen 
		\end{itemize}		
	\end{frame}

	\begin{frame}{Homepages}
		\begin{itemize}
			\item Vorlesungshomepage: \texttt{http://algo2.iti.kit.edu/AlgorithmenI2014.php}
			\begin{itemize}
				\item Interessante Organisatorische Details
				\item Vorlesungsfolien
				\item Die Übungsblätter
				\item Prof. Sanders Buch
				\item Literaturliste
			\end{itemize}
			\item Meine Folien: \texttt{https://github.com/Strandtasche/AlgoTutSS2014}
			\item ILIAS: \texttt{https://ilias.studium.kit.edu}
			\item Fakultät für Informatik $\rightarrow$ SS 2014 $\rightarrow$ Algorithmen I mit Übung			
		\end{itemize}		
	\end{frame}

	\begin{frame}{Übungsblätter}
		Das meiste wurde schon in der Vorlesung gesagt:
		\begin{itemize}
			\item 9 Tage Bearbeitungszeit: von Mittwoch bis Freitag der nächsten Woche
			\item Abgabe 12:45 Uhr im Kasten hier vor dem Raum (-118)
			\item 2er-Teams sind erlaubt
			\item Teamwechsel im Semester sind nicht vorgesehen, können aber in Sonderfälle mit Rücksprache durchgeführt werden
			\item Wichtig Wichtig: Tutoriumsnummer rechts oben groß auf die Abgabe!
			\item Kein Verpflichtender Übungsschein sondern Klausurbonus von 1/2/3 Punkten
			\item Zum ersten Mal dieses Jahr: Programmieraufgaben
			\begin{itemize}
				\item 4 Stück, über das Semester verteilt, in Java
				\item keine Klassenstrukturen sondern tatsächlich nur der Algorithmus
				\item Praktomat Abgabe: Ohne Verpflichtenden Code Style
				\item Dazu mehr wenn die erste Aufgabe kommt...
			\end{itemize}
		\end{itemize}		
	\end{frame}

\section{Pseudocode und Schleifeninvarianten}
	\begin{frame} {Pseudocode}
		
	\end{frame} 

\end{document}
