%% LaTeX-Beamer template for KIT design
%% by Erik Burger, Christian Hammer
%% title picture by Klaus Krogmann
%%
%% version 2.1
%%
%% mostly compatible to KIT corporate design v2.0
%% http://intranet.kit.edu/gestaltungsrichtlinien.php
%%
%% Problems, bugs and comments to
%% burger@kit.edu

\documentclass[18pt]{beamer}
\usepackage[utf8]{inputenc}
%% SLIDE FORMAT

% use 'beamerthemekit' for standard 4:3 ratio
% for widescreen slides (16:9), use 'beamerthemekitwide'

\usepackage{templates/beamerthemekit}
% \usepackage{templates/beamerthemekitwide}

%% TITLE PICTURE

% if a custom picture is to be used on the title page, copy it into the 'logos'
% directory, in the line below, replace 'mypicture' with the 
% filename (without extension) and uncomment the following line
% (picture proportions: 63 : 20 for standard, 169 : 40 for wide
% *.eps format if you use latex+dvips+ps2pdf, 
% *.jpg/*.png/*.pdf if you use pdflatex)

%\titleimage{mypicture}

%% TITLE LOGO

% for a custom logo on the front page, copy your file into the 'logos'
% directory, insert the filename in the line below and uncomment it

%\titlelogo{mylogo}

% (*.eps format if you use latex+dvips+ps2pdf,
% *.jpg/*.png/*.pdf if you use pdflatex)

%% TikZ INTEGRATION

% use these packages for PCM symbols and UML classes
% \usepackage{templates/tikzkit}
% \usepackage{templates/tikzuml}

% the presentation starts here

\title[Croggle Abschlusspräsentation]{Croggle}
\subtitle{Abschlusspräsentation}
\author{Lukas Böhm}

\institute{Praxis der Softwareentwicklung Wintersemester 2013/2014}

\begin{document}

% change the following line to "ngerman" for German style date and logos
\selectlanguage{ngerman}

%title page
\begin{frame}
\titlepage
\end{frame}

%table of contents
%~ \begin{frame}[allowframebreaks]{Übersicht}
%~ \tableofcontents
%~ \end{frame}

\section{Einführung}
	\subsection{Croggle: Motivation}
	\begin{frame}{Was ist Croggle?}
		\begin{figure}
			\includegraphics[width=4cm]{logos/ic_launcher.pdf}
		\end{figure}
	\end{frame}

	\begin{frame}[<+->]{Was ist Croggle?}
		\begin{itemize}
			\item Lernanwendung
			\item Für Kinder ab 8 Jahren
			\item Lauffähig auf Android ab spätestens Version 4.0.4
			\item Zum Erlernen des $\lambda$-Kalküls
		\end{itemize}
		Hauptziel: Kinder früh an Informatik bzw. die Denkweise der (funktionalen) Programmierung heranführen
	\end{frame}

	\subsection{Einführung $\lambda$-Kalkül}
	\begin{frame}[<+->]{Das $\lambda$-Kalkül}
		Was ist das?
		\begin{itemize}
			\item Für uns vergleichbar mit Grammatiken oder Automaten 
				\\ $\rightarrow$ Buchstabenfolgen produzieren
			\item Vorteil: Alles in einem einzigen Term. Keine Alphabetsdefinitionen etc.
			\item Turingvollständig
		\end{itemize}
	\end{frame}
	
	\begin{frame}{Das $\lambda$-Kalkül}
		Drei Komponenten:
		\pause
		\begin{block}{Literale}
			Einfache Buchstaben: a, b, c\dots \\
			Heißen eigentlich "`ungebundene Variablen"'
		\end{block}
		\pause
		\begin{block}{Variablen}
			Auch einfache Buchstaben: a, b, c, x, y, z...\\
			$\rightarrow$ Muss aus Kontext ersichtlich sein, ob Zeichen Variable oder Literal
		\end{block}
		\pause
		\begin{block}{Funktionen}
			Auch "`Abstraktionen"'
			\begin{itemize}
				\item Einleitung durch $\lambda$
				\item Nehmen genau 1 Argument
				\item Name des Arguments (Variable) nach $\lambda$
				\item Anschließend Punkt und Funktionsdefinition
			\end{itemize}
		\end{block}
	\end{frame}

	\begin{frame}{Das $\lambda$-Kalkül}
		Dazu kommt noch gewöhnliche Klammerung (runde Klammern).\\
		Starke Parallelen zur Mathematik:
		\begin{exampleblock}{Funktionsdefinition}
			\begin{description}
				\item[Mathematik:] \textless Funktionsname\textgreater:\textless Argumentname\textgreater $\mapsto$ \textless Funktionskörper\textgreater
				\item[Lambdakalkül:] $\lambda$\textless Argumentname\textgreater.\textless Funktionskörper\textgreater
			\end{description}
		\end{exampleblock}
		\pause
		\begin{alertblock}{Der Clou}
			Elemente, die hinter Abstraktionen geschrieben werden, sind automatisch Argumente einer Applikationen \\
			$\rightarrow $ Äquivalenzumformungen wie beim Auswerten eines Programms
		 \end{alertblock}
	\end{frame}

	\begin{frame}{Beispiele}
		\begin{exampleblock}{Identität}
			\begin{description}
				\item[Mathematik:] $Id: x \mapsto x$
				\item[Lambdakalkül:] $\lambda x.x$
			\end{description}
			$(\lambda x.x) a \rightarrow a$
		\end{exampleblock}
		\pause
		\begin{exampleblock}{"`Quadrat"'}
			\begin{description}
				\item[Mathematik:] $Sq: x \mapsto x x$
				\item[Lambdakalkül:] $\lambda x.x x$
			\end{description}
		\end{exampleblock}
		\pause
		\begin{exampleblock}{Endlosschleife}
			\begin{description}
				\item[Lambdakalkül:] $(\lambda x.x x)(\lambda x.x x)$
			\end{description}
		\end{exampleblock}
	\end{frame}

	\begin{frame}{Spielidee}
		\center{Wie macht man daraus ein unterhaltsames Spiel für Kinder?}
	\end{frame}

	\subsection{Spielidee}
	\begin{frame}[<+->]{Alligator Eggs!}
		Spielidee von Bret Victor: \textcolor{blue}{\href{/http://worrydream.com/AlligatorEggs}{worrydream.com/AlligatorEggs}}
		\begin{block}{Prinzip}
			\begin{itemize}
				\item Stelle Lambdaterme durch farbige Alligatoren und ihre Eier dar
				%~ \item Die Eier sind Literale oder Variablen
				%~ \item Die Farben sind ihre Namen
				\item 2D statt 1D
				%~ \item Funktionskörper werden vertikal aufgetragen
				%~ \item Applikationen werden horizontal nebeneinander dargestellt
				\item Auswertung geschieht durch "`Fressregel"'
				%~ \item Funktionsanwendungen sind fressende Alligatoren
				\item $\rightarrow$ Gefressenes schlüpft aus den gleich gefärbten Eiern unter dem Fresser
			\end{itemize}
		\end{block}
	\end{frame}

	\begin{frame}{Beispiel}
		\begin{figure}
				$(\lambda x.x x)(\lambda y.y) z$ 
				$\;\Leftrightarrow\;$
				$\vcenter{\hbox{\includegraphics[width=4cm,type=pdf,ext=.pdf,read=.pdf]{logos/(Ax.xx)(Ay.y)z}}}$
		\end{figure}
		\vspace{-1cm}
		\pause
		\begin{figure}
				$(\lambda y.y)(\lambda y.y) z$ 
				$\;\Leftrightarrow\;$
				\visible<2->{$\vcenter{\hbox{\includegraphics[width=4cm,type=pdf,ext=.pdf,read=.pdf]{logos/(Ay.y)(Ay.y)z}}}$}
		\end{figure}
		\vspace{-1cm}
		\pause
		\begin{figure}
				$(\lambda y.y) z$ 
				$\;\Leftrightarrow\;$
				\visible<3->{$\vcenter{\hbox{\includegraphics[width=3.5cm,type=pdf,ext=.pdf,read=.pdf]{logos/(Ay.y)z}}}$}
		\end{figure}
		\vspace{-1cm}
		\pause
		\begin{figure}
				$z$ 
				$\;\Leftrightarrow\;$
				\visible<4>{$\vcenter{\hbox{\includegraphics[width=2cm,type=pdf,ext=.pdf,read=.pdf]{logos/z}}}$}
		\end{figure}
		\vspace{-1cm}
		\vfill
		\pause
		$\rightarrow$ Kindgerecht, aber ohne Zielstellung langweilig
	\end{frame}

\section{Produktvorstellung}
	\subsection{Croggle, die App}
	\begin{frame}{Croggle in Aktion}
		\center{Wie löst Croggle dieses Problem?}
	\end{frame}
	
	\subsection{Leveltypen}
	\begin{frame}[<+->]{Spielmechanik in Croggle}
		Drei verschiedene Spielmodi:
		\begin{itemize}
			\item Multiple Choice
			\item Vorgegebene Lücken füllen
			\item Terme ergänzen
		\end{itemize}
	\end{frame}

	\subsection{Zusatzfunktionen}
	\begin{frame}[<+->]{Weitere Features}
		\begin{itemize}
			\item Mehrere Benutzer
			\item Statistiken
			\item Achievements
			\item Farbenblindenmodus
			\item Mehrsprachigkeit
		\end{itemize}
		und vieles mehr!
	\end{frame}

\section{Umsetzung}
	\begin{frame}{Realisierung}
		\center{Wie haben wir das geschafft?}
	\end{frame}

	\subsection{Planung \& Entwurf}
	\begin{frame}[<+->]{Tools}
		\begin{block}{Planung}
			\begin{itemize}
				\item LaTeX für alle Dokumente
				\item Gimp für erste Screen Mockups
				\item ArgoUML für das UML der Planungsphase
			\end{itemize}
		\end{block}
		\begin{block}{Entwurf}
			\begin{itemize}
				\item UMLet für Sequenzdiagramme der Entwurfsphase
				\item Nichts drückt den Inhalt eines Softwareprojekts besser aus als sein Quellcode \\
				$\rightarrow$ Java Quellcode bereits für den Entwurf
				\item KIT hauseigenes TexDoclet zur Generierung der Klassendokumentation aus Java Code
				\item UMLGraph zur Erstellung von Klassendiagrammen aus Java Quellcode
			\end{itemize}
		\end{block}
	\end{frame}

	\subsection{Implementierung \& Test}
	\begin{frame}[<+->]{Tools}
		\begin{block}{Implementierung}
			\begin{itemize}
				\item Eclipse IDE
				\item Android SDK rev19
				\item libGdx Framework
				%~ \item Nuzung von libGdx' TexturePacker
				\item git zur Revisionskontrolle
				\item GitHub: \textcolor{blue}{\href{http://github.com/TeamCroggle}{github.com/TeamCroggle}} \\
					$\rightarrow$ Open Source
			\end{itemize}
		\end{block}
		\begin{block}{Testphase}
			\begin{itemize}
				\item jUnit 3 (Version 4 von Android nicht unterstützt)
				\item Emma, einschließlich Eclipse Integration EclEmma
				\item MonkeyRunner und Monkey Tool aus dem Android SDK
			\end{itemize}
		\end{block}
	\end{frame}
	
	\begin{frame}[<+->]{Nebenprodukt Desktopversion}
		\begin{itemize}
			\item Bereits in der Demo gezeigt
			\item Enstanden zwischen Implementierung und Test
			\item Durch Framework einfach gemacht
			\item Teilweise schon während Implementierung auf Portabilität geachtet
			\item Hauptarbeit SQLite
		\end{itemize}
	\end{frame}

	\subsection{Problemstellungen}
	\begin{frame}[<+->]{Herausforderungen}
		Die größten Hürden, die es zu bewältigen galt: 
		\begin{itemize}
			\item Einarbeitung ins $\lambda$-Kalkül
			\item Einarbeitung Android bzw. libGdx
			\item Darstellung der Alligatoren getrennt nach MVC
			\item Eigenes auf JSON aufbauendes Format für Level und Tutorials
			\item Speicherung der Nutzerdaten in Sqlite
			\item Und natürlich der ganze Rest der App
			%~ \item Eigene Abstraktion der Android Lokalisierungsdienste für Plattformunabhängigkeit
		\end{itemize}
	\end{frame}

	%~ \begin{frame}{Probleme}
		%~ \begin{itemize}
			%~ \item Verschiedene Entwicklungsumbebungen und Softwareversionen
			%~ \pause
			%~ \item Bug in libGdx $\rightarrow$ Bug Report
			%~ \pause
			%~ \item Keine WYSIWYG Editoren für die GUI
		%~ \end{itemize}
	%~ \end{frame}

\section{Rezeption \& Fazit}
	\subsection{Meinungen von Außerhalb}
	\begin{frame}[<+->]{Meinungen von Außerhalb}
		Das meinen die Tester:
		\begin{itemize}
			\item Tolle Grafiken
			\item Hier und da müssten zusätzliche Erklärungen hin
			\item Wirkt sehr professionell
			\item Analogie Alligator $\Leftrightarrow$ Lambdakalkül hinkt/verwirrt
		\end{itemize}
	\end{frame}

	\subsection{Eigene Meinung}
	\begin{frame}[<+->]{Unsere Meinung}
		Das meinen wir:
		\begin{itemize}
			\item Tolles Projekt: Viele Freiheiten ohne aber überhaupt erst mal ein Konzept finden zu müssen
			\item Sehr viel gelernt, sowohl technisch wie auch bzgl. Teamarbeit
			\item Mehr Wagemut zum Abändern der Spielidee, damit es logischer wird, wäre gut gewesen
			\item Gute Balance gefunden zwischen Menge an Features und insgesamter Ausgereiftheit
			\item Ein bisschen zu viel vorgenommen $\rightarrow$ wenig Freizeit
		\end{itemize}
	\end{frame}

	%~ \subsection{Lehren aus dem Projekt}
	%~ \begin{frame}{Was wir gelernt haben}
	%~ \end{frame}

\section{Ende}
	\begin{frame}{Ende}
		\center{Vielen Dank für Ihre Aufmerksamkeit}
	\end{frame}

	%~ \subsection{Amüsantes}
	%~ \begin{frame}{Bloopers}
		%~ \begin{itemize}
			%~ \item Für den Entwurf wurde das KIT TexDoclet von uns geforkt und so stark modifiziert, dass mir vorgeschlagen wurde, Maintainer zu werden
			%~ \item Inhaltsverzeichnis ohne latexmk kaputt
			%~ \item Unübersichtliches Klassendiagramm dank Autogenerierung
			%~ \item Umstellung Repository kurz vor der Testphase
		%~ \end{itemize}
	%~ \end{frame}
%~ 
	%~ \begin{frame}{Miscellaneous}
	%~ Für weitere Stichpunkte
		%~ \begin{itemize}
			%~ \item Statistiken und genaue Zahlen über weiterverwendbaren Code
			%~ \item Statistiken hier einfügen
			%~ \item Zielsetzung: Altersgruppe, Geräteanforderungen, Langzeitmotivation...
		%~ \end{itemize}
	%~ \end{frame}

\end{document}
